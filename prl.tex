\documentclass[a4paper,%            A4 papir size
               aps,%                Document gets APS layout
               prl,%                more layout, section numbering is removed
               amsfonts,%           load AMS fonts
               amssymb,%            load more AMS symboler
               amsmath,%            load AMS math (equation environment and so on...)
               nobibnotes,%         If BibTex is not used, REVTeX should be told to use normal footnotes
               twocolumn, %         two-column layout
               twoside,%            to-sided print
               balancelastpage,%    balancing the last page when using two columns
               eqsecnum] %          numbering the equations with the sectionnumber
               {revtex4-1}
\usepackage[utf8]{inputenc} %       Correct formatting, makes danish letters available
\usepackage[english]{babel} %       Language

% Place all outher packages you use here i.e. graphicx, subfiles...
\usepackage{amsmath, amssymb}
\usepackage{url}
\usepackage{wrapfig}
\usepackage{subfiles}
\usepackage[squaren]{SIunits}
\usepackage{caption}
\captionsetup{labelfont={footnotesize,bf},textfont=footnotesize,format=hang,indention=-0.5cm} %makes nicer indents
\usepackage[normalem]{ulem}
\usepackage{subfigure}
\usepackage{lipsum}
\usepackage{todonotes}

\begin{document}                                % Starts the actual document
\selectlanguage{english}

\title{Writing articles for Applied statistics \\ a laTeX template} % Title of the document, \\ can be used to generate linebreaks
\date{November 17, 2016}                                                % Date of the document, \today can be used

% Author on the articles. Add a new line of \author{Name} for each author
\author{Niccolo Maffezzoli}
\author{Daniel S.\ Nielsen}
\author{Troels C.\ Petersen}
\affiliation{Niels Bohr Institute - Applied statistics - Project 1 group X}   % The authors affiliation

% With REVTeX the abstract must be before \maketitle 
\begin{abstract}                                              
The abstract should be a short description of what your article includes, such that the reader can
determine, if the article is relevant/interesting. Also, key results and possibly conclusions should
be reported here. The abstract should typically not be longer than 8-10 lines.
\end{abstract}

\maketitle 

% Resets the counters so references are logical:
\setcounter{section}{1} %Sets the sectioncounter to 1
\setcounter{equation}{0} %Sets the equationcounter to 0



% Begin the sections
% ------------------------------------------------------------------------------------------ %
\section{Introduction}
% ------------------------------------------------------------------------------------------ %
The two following equations should make it easier for you to obtain expressions for $g$ and start taking measurements.
In the first experiemnt you will be measuring the period of a pendulum with the length $L$ and calculate $g$ by using the pendulum equation:
\begin{equation}
T = 2\pi\sqrt{\frac{L}{g}} \label{eq:Pendul}
\end{equation}
This means that $g$ can be determined as:
\begin{equation}
g = L \left( \frac{2\pi}{T} \right)^2 \label{eq:Pendul}
\end{equation}

The second experiment is a ball rolling down an inclined track. Through the laws of energy conservation and the assumption that the
acceleration is constant, $g$ can be found to be:
\begin{equation}
g = \frac{a}{\sin(\theta + \Delta\theta)}\left[1+\frac{2}{5}\frac{R^2}{R^2 - \left(\frac{d}{2}\right)^2}\right] \label{eq:Ball}
\end{equation}
where $a$ is the measured acceleration of the center of mass (measured using the distance and time between the light gates), $\theta$ is the angle of the incline with respect to the table, $\Delta \theta$ is the angle of the table, $R$ is the radius of the ball and $d$ is the width of the track it is rolling on. To make figures, see Fig.\ \ref{fig:fit}.


% ------------------------------------------------------------------------------------------ %
\section{Method}
\subsection{Pendulum}
\lipsum[1-3]

\subsection{Ball on an incline}
\lipsum[1-3]

\begin{figure}[h]
\centering
\includegraphics[width=\linewidth]{NiceFigureExample.pdf}
\caption{This is a very nice figure. But the data and subsequent fit looks a bit dubious. \label{fig:fit}}
\end{figure}


% ------------------------------------------------------------------------------------------ %
\section{Results}
\subsection{Pendulum}

\begin{table}[h!]
  \centering
  \begin{tabular}{lrrr}
    \hline
    \hline
Variable	&Value	&Statistical error	&Systematic error\\
\hline
\quad	Experiment 1 \\
\hline
    Period $T$	&8.1231 s	&0.0942 s	&0.060 $\usk\meter\per\second\squared$\\
    Length $L$	&8.1231 m	&0.0942 m	&0.060 $\usk\meter\per\second\squared$\\
    Gravity $g$	&9.887	&0.005 $\usk\meter\per\second\squared$ &\\
\hline
\quad	Experiment 2 \\
\hline 
    Period $T$	&8.8712 s	&0.0123 s	&0.060 $\usk\meter\per\second\squared$\\ 
    Length $L$	&8.5812 m	&0.0123 m	&0.060 $\usk\meter\per\second\squared$\\
    Gravity $g$	&9.9393	&0.006 $\usk\meter\per\second\squared$ &\\    
    \hline
    \textbf{Resulting $g$}	&\textbf{9.421}	&\textbf{0.03 $\usk\meter\per\second\squared$} &\\
    \hline
    \hline
  \end{tabular}
  \caption{Different variables and their uncertainty from the pendulum experiment used to determine $g$, and their impact on $g$. \label{tab:GPen}}
\end{table}


\subsection{Ball on an incline}
\lipsum[1-3]


% ------------------------------------------------------------------------------------------ %
\section{Discussion}


% ------------------------------------------------------------------------------------------ %
\section{Conclusion}



% ------------------------------------------------------------------------------------------ %
\begin{thebibliography}{99}  
% Literature you can refer to
\bibitem{Griffiths} D. J. Griffiths, \emph{Introduction to Electrodynamics 3. edition}, (Pearson Education, Inc. 1999)                                 
\end{thebibliography}

\end{document}                                                % Slut på dokument, alt herefter ignoreres
